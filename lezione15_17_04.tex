
Un elemento fondamentale dell'OATS è la presenza di una tavola rotante, necessaria
alla misurazione dell'emissione per ogni angolo del dispositivo, mantenendo
fisso il punto di ricezione. Nel caso in cui ciò non sia possibile,
si fa ruotare l'antenna attorno all'EUT a distanza $R$ e l'OATS deve
avere un raggio $\geq 1,5R$.

È evidente che l'OATS influenza la propagazione del segnale e per questo motivo
è necessaria la sua \textbf{taratura}, va confrontato ogni OATS con il modello ideale.

Il primo passo necessario alla taratura è la misura di $V_{\text{DIRECT}}$ ossia la tensione
diretta tra il generatore e il misuratore, collegati mediante un cavo.
Poi si effettua la misura di $V_{\text{SITE}}$ ossia 
$$
V_{\text{SITE}} = V_L -\alpha_T - \text{SA} - \alpha_r = V_{\text{DIRECT}} - \text{SA}
$$

SA è proprio l'attenuazione di sito, al quale vanno sottratti i fattori di antenna e il
fattore di accoppiamento per ottenere il \textit{Normalised Site Attenuation} (NSA)
$$
V_{\text{DIRECT}} - V_{\text{SITE}} - \text{AF}_T - \text{AF}_R - \Delta\text{AF} = \text{NSA}
$$

Per avere un'analisi completa, l'antenna ricevente deve misurare le emissioni anche ad altezze
e inclinazioni differenti, ciò comporta una variazione di distanza percorsa dai raggi diretti e quelli riflessi,
è quindi logico attendersi un valore di tensione differente.
Il $V_{\text{SITE}}$ da misurare deve quindi essere il massimo per ogni altezza dell'antenna ricevente $h_1$,
tipicamente tra \SI{1}{\meter} e \SI{4}{\meter} per misure a \SI{10}{\meter} e tra \SI{3}{\meter} 
e \SI{6}{\meter} per misure a \SI{30}{\meter}.

Infine l'NSA ottenuto (\textit{sperimentale}) viene confrontato con quello teorico e deve essere pari
ad esso $\pm\SI{4}{\decibel}$ (vedi tabelle E1 E2 ed E3).

Gli NSA variano a seconda della polarizzazione delle antenne, verticale ed orizzontale,
le antenne utilizzate per effettuare la misura sono di due tipi: dipoli accordati ($\lambda/2$) 
o antenne log-periodiche a banda larga. Le antenne a banda larga non hanno le stesse prestazioni dei
dipoli accordati ma c'è un enorme vantaggio in termini di tempo, dato che non si richiede la sostituzione
della stessa per ogni valore di frequenza.

I fattori di antenna sono caratteristiche note ai costruttori, il $\Delta\text{AF}$ invece è diverso da 0
solo per dipoli a $\lambda/2$ e distanza pari a \SI{3}{\meter} (tabella E4).

Con la polarizzazione verticale è importante mantenere sempre almeno \SI{25}{\centi\meter} di
distanza tra l'antenna e il piano di massa per evitare accoppiamenti con lo stesso.
%0:58

Una soluzione pratica è quella di utilizzare un sistema che permetta di variare l'altezza dell'antenna con continuità
utilizzando un Max Hold per rilevare la massima ampiezza di campo mentre l'antenna si sposta, se lo sweep delle
frequenze avviene in maniera abbastanza veloce da poter considerare ogni sweppata alla stessa altezza, 
solo l'analizzatore di spettro è in grado di eseguire una misura simile.

Solitamente viene eseguita prima la misura con polarizzazione orizzontale, quella con polarizzazione
verticale è più sensibile, se già con quella meno sensibile si ottengono valori fuori limite,
significa che il sito ha grossi problemi di attenuazione.
\newpage




Eventuali criticità relative alla misura:
\begin{itemize}
 \item Errore grossolano tipo distanza di misura e altezze delle antenne
 \item Verificare la correttezza dei fattori di attenuazione
 \item Avere la certezza che non ci sia stata una variazione del funzionamento degli strumenti (drift termico)
\end{itemize}
Se le precedenti soluzioni non portano ad un miglioramento dei valori, allora il problema
starà proprio nel sito che stiamo misurando, come
\begin{itemize}
 \item Piano di terra con non perfetta conducibilità
 \item Discontinuità dei pannelli dovute ad errati fissaggi, distanze troppo elevate con probabili risonanze
 \item Discontinuità dovute alla sporgenza delle viti o alla loro assenza (fori liberi)
 \item Problemi dovuti alle tendostrutture non perfettamente permeabili ai campi elettromagnetici
\end{itemize}
(tabella pagina 145)

Nel metodo di misura discreto, si ha un'incertezza maggiore sull'attenuatore, il dispositivo
necessario a misurare la tensione diretta, nel metodo con analizzatore di spettro, invece, l'incertezza è di
circa \SI{1,6}{\decibel} includendo sia l'attenuatore che il voltmetro.
