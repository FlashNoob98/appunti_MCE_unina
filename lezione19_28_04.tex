\paragraph{Comportamento alle correnti differenziali e di modo comune}
Il filtro a $\pi$ appena presentato si comporta diversamente se analizzato
secondo le correnti di modo comune oppure differenziale.

Il filtro induttivo è un circuito magnetico in cui vengono avvolti i due 
cavi di fase e neutro, avvolti nello stesso verso,
ciò comporta sicuramente che $L_P = L_N$, si ha inoltre una mutua induttanza
dovuta all'influenza reciproca dei flussi indotti da uno dei due avvolgimenti, 
che causa una tensione indotta nell'altro. Per come sono fatti gli
avvolgimenti, quasi tutto il flusso si accoppia, ovvero il fattore di accoppiamento 
è quasi unitario, ossia $M\simeq L$. Questo dispositivo si chiama 
\textit{soppressore di modo comune}.

\paragraph{Correnti di modo comune} Ossia $I_P = I_N = I_C$.
%disegna filtro 0:30:00
Il circuito del filtro è perfettamente simmetrico ed in questo caso
anche il verso delle correnti è simmetrico, ciò implica che $V_P$ e $V_N$ hanno
lo stesso valore, ossia su $C_D$ non ci sarà differenza di potenziale e quindi non
vi sarà alcuna corrente. 

Si può pensare di separare il circuito di fase da quello di terra
a causa della simmetria del circuito, tralasciando momentaneamente il cavo di terra
in comune e l'accoppiamento magnetico tra i due circuiti.

\begin{figure}[h] %circuito singola fase parziale
 \centering
 \begin{circuitikz}
 
 \end{circuitikz}
\end{figure}

La tensione ai capi dell'induttanza di ``green wire''  $L_{\text{GW}}$ sarà
pari a $V = I_C j \omega 2L_{\text{GW}}$, per imporre la stessa tensione 
del circuito completo, si raddoppia l'induttanza.

Per quanto riguarda la mutua induttanza invece è possibile aggiungere una tensione
pari a $V = I_C j \omega L + I_C j \omega M = I_C j \omega (L+M)$ dato che le correnti
$I_C$ sono nello stesso verso, per tenere conto di questo fenomeno si può quindi
modificare il valore dell'induttanza in $L+M$.

Analizzando l'ultima maglia si vede che la corrente preferirà percorrere nel 
condensatore $C_C$ se $\frac{1}{2 \pi f C_C} < \SI{50}{\ohm} 
\Rightarrow f > \SI{1.5}{\mega\hertz}$ con $C_C$ pari a \SI{2200}{\pico\farad} come 
precedentemente supposto.

\paragraph{Correnti di modo differenziale} Ossia $I_P = I_D = - I_N$.
Il circuito è ancora perfettamente simmetrico per quanto riguarda i componenti
ma totalmente antisimmetrico per quanto riguarda le correnti, ciò implica che 
la tensione $V_P$ sarà pari a $-V_N$, dovrà esistere quindi un punto virtuale a tensione
nulla, il condensatore $C_D$ si può sostituire con 2 condensatori in serie
con una capacità pari a $2C_D$.
%0:53
In questo modo il conduttore di protezione non sarà attraversato da alcuna corrente, si 
può rimuovere l'induttanza di ``green wire''.

Il contributo della mutua induttanza sarà differente a causa del diverso verso di corrente
tra i due cavi, l'induttanza si trasformerà nel valore $L-M$.
%circuito equivalente 1:01:30
Ricordando che $L$ ed $M$ sono valori prossimi tra loro, si vede che per le correnti
differenziali, il dispositivo non oppone una forte impedenza, idealmente sarebbe nulla.
La non idealità della struttura, permette di attenuare leggermente anche le correnti
di modo differenziale.

A causa del differente valore dei condensatori $C_C$ e $2C_D$ si vede che il parallelo
tra i due condensatori è approssimabile alla presenza del solo condensatore $2C_D$.
Le frequenze attenuate dai condensatori sono quelle di valore superiore a
$\frac{1}{2 \pi f 2C_D}< \SI{50}{\ohm} \Rightarrow f > \SI{34}{\kilo\hertz} $

L'induttanza di ``green wire'' viene realizzata avvolgendo il cavo di terra attorno
ad un toro ferromagnetico, il terminale viene crimpato ed avvitato sulla scheda
o sulla massa del dispositivo. Le saldature di un'eventuale induttanza renderebbero
meno affidabile la continuità elettrica del cavo di protezione.
