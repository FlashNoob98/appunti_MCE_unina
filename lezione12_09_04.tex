
Se avessimo una scheda elettronica con un generatore ed un carico,
si avrebbero due conduttori, i quali sarebbero attraversati dalla corrente
differenziale, si può supporre che si sia anche una corrente
di modo comune aggiuntiva, dovuta a causa della non idealità dell'ossido,
che si richiuda verso la massa al di sotto della scheda.

$$
\begin{cases}
I_1 = I_C + I_D \\
I_2 = I_C - I_D
\end{cases}
\Rightarrow \quad
\begin{cases}
I_C = \frac{I_1+I_2}{2} \\
I_D = \frac{I_1-I_2}{2}
\end{cases}
$$

Immaginando di avere 2 fili di corrente a distanza $S/2$ dall'origine,
le due correnti procedono nello stesso verso, verso l'asse $Z$.
Per caratterizzare il campo generato possiamo scomporne le componenti lungo i tre assi
ottenendo in ogni punto dello spazio un campo elettrico $E_\theta(r,\theta,\varphi)$.
A grande distanza si vede che il campo elettrico dipende solo dalla variabile $\theta$
$$
 E_\theta = M \cdot I \cdot \frac{e^{-j\beta_0 r}}{r} F(\theta)
$$

$M$ e $F(\theta)$ sono due grandezze che dipendono dal tipo di dipolo considerato, $\beta_0 = 2\pi/\lambda$.

\paragraph{Dipolo Hertziano} È un dipolo \textit{piccolo elettricamente}
ossia piccolo rispetto alla lunghezza d'onda che lo attraversa, se così
non fosse non si potrebbe affermare che il valore della corrente sia
uniforme lungo la sua lunghezza.

I parametri del dipolo hertziano sono i seguenti:
$$
\begin{cases}
M =  2\pi j \cdot 10^{-7} L f \\
F(\theta) = \sin \theta
\end{cases}
$$

I raggi che collegano i due conduttori con il punto di osservazione si possono supporre paralleli
a causa della grande di stanza del punto di osservazione dal dipolo.
Il campo totale $E_T$ sarà pari, applicando il principio di sovrapposizione degli effetti:
$$
 E_T = E_1 + E_2 = M\cdot F(\theta)\cdot \left[\frac{I_1 e^{-j\beta_0 r_1}}{r_1} +\frac{I_2 e^{-j\beta_0 r_2}}{r_2}  \right]
$$

%%0:33 inserisci immagine fili a grande distanza r1 r e r2 cap 9 libro

$\Delta$ è il cammino addizionale rispetto al conduttore più vicino
che il campo deve percorrere per raggiungere il punto P,
$\gamma$ è l'angolo tra la retta che congiunge P all'origine e l'asse Y, di conseguenza
$\Delta$ sarà 
$$
\Delta = \frac{s}{2}\cos\gamma = \frac{s}{2}\cdot \hat{\imath}_y\cdot \hat{\imath}_r
$$

Un altro modo per esprimere $\Delta$ è quello di proiettare prima il punto sul piano XY, successivamente
proiettare il punto ottenuto $\Delta'$ sull'asse Y.
$$
\cos\gamma = \cos\left(\frac{\pi}{2} - \theta\right) \cdot \sin\varphi = \sin\theta \cdot \sin\varphi
$$

Nel complesso
$$
\begin{cases}
r_1 = r + \Delta = r + \frac{s}{2} \sin\theta \sin\varphi \\
r_2 = r - \Delta = r - \frac{s}{2} \sin\theta \sin\varphi
\end{cases}
$$

$$
E_T = 2 \pi j \cdot 10^{-7} L f \frac{e^{-j \beta_0 r}}{r} \left[I_1 e^{-j \beta_0 \Delta} + I_2 e^{j \beta_0 \Delta} \right]\sin\theta
$$

\paragraph{Calcolo campo con correnti di modo differenziale}
In questo caso $I_1 = I_D$ e $I_2 = -I_D$
\begin{equation*}
 \begin{split}
E_D & = 2 \pi j \cdot 10^{-7} L f \frac{I_D}{r}e^{-j \beta_0 r} \left[e^{-j \beta_0 \Delta} - e^{j \beta_0 \Delta} \right]\sin\theta \\
 & = 2 \pi j \cdot 10^{-7} L f \frac{I_D}{r}e^{-j \beta_0 r} \left[-2j\sin(\beta_0\Delta)\right]\sin\theta \\ %%e il meno???
 & = 4\pi\cdot10^{-7}Lf\frac{I_D}{r}e^{-j\beta_0 r} \sin\left(\beta_0 \frac{s}{2} \sin\theta \sin\varphi \right)\sin\theta
  \end{split}
\end{equation*}

Valutando $\left|E_D\right|$ per cercarne il massimo si ottiene
\begin{equation*}
 \begin{split}
\text{max}&\left|E_D\right| = 4\pi\cdot10^{-7}Lf\frac{I_D}{r}\sin\left(\beta_0 \frac{s}{2} \sin\theta \sin\varphi \right)\sin\theta \\
& \stackrel{\theta = \pi/2}{=} 4\pi\cdot10^{-7}Lf\frac{I_D}{r}\sin\left(\frac{2\pi}{\lambda} \frac{s}{2} \sin\varphi \right) 
  \end{split}
\end{equation*}

Possiamo affermare quindi che il valore massimo del campo si trova sul piano XY, ossia quando
$\theta$ è pari a $\pi/2$.
Analizziamo ora la funzione rispetto a $\varphi$, se $s\ll\lambda$ ossia i due conduttori
si trovano ad una distanza molto inferiore rispetto alla lunghezza d'onda, il seno presente nell'equazione
si confonderà con l'argomento del seno stesso:
$$
\left|E_D\right| \stackrel{s\ll\lambda}{=} \frac{4\pi^2\cdot10^{-7}}{r}f\frac{I_D L}{\lambda}s  
$$
In conclusione
$$
\left|E_D \right|_{\text{max}} \stackrel{\theta,\varphi = \pi/2}{=} \frac{4\pi^2\cdot10^{-7}I_D}{c\cdot r}Lsf^2
$$
Se ci si sposta lungo l'asse X, il campo è pari a 0 perché i percorsi dei differenti campi sono identici,
ma il loro valore è uguale e opposto.

