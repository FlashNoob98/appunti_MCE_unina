
Un filtro con risposta in frequenza molto stretta, richiede una risposta nel tempo
molto ampia, ossia richiede molto tempo per andare a regime.

Fissata una RBW del filtro intermedio a \SI{3}{\decibel}, si avrà in uscita
un segnale composto dalla $f_{\text{if}}$ e da alcune componenti laterali, ossia
un segnale modulato in ampiezza la cui portante è proprio la $f_{\text{if}}$.
L'unione dei picchi nel tempo del segnale viene chiamato \textbf{inviluppo} del segnale,
è in realtà la parte ``interessante'' del segnale che viene infatti inviato
al \textit{rivelatore di inviluppo} che trasforma il segnale modulato nel suo inviluppo.

\begin{figure}[h] %ricevitore di inviluppo
\centering
 \begin{circuitikz}[american voltages]
 \draw
 (0,2) to [full diode,o-] (2,2)
       to [resistor] (2,0)
 (0,0) to [short, o-] (2,0)
 ;
 \draw
 (-0.5,2) to [open, v=$V_{\text{in}}$] (-0.5,0)
 (2.6,2) to [open, v^>=$V_{\text{out}}$] (2.6,0)
 ;
 \end{circuitikz}
 \caption{Rivelatore di inviluppo}
\end{figure}

Quando si ricostruisce il contenuto spettrale del segnale, viene riportato il
valore efficace dell'ampiezza del segnale ad ogni singola frequenza di interesse,
in realtà il valore visualizzato è la somma dei contributi dati dalla componente
in quella frequenza e dalle frequenze \textit{laterali} che rientrano nella 
\textit{Resolution BandWidth}.

Il primo parametro importante da determinare nell'analisi di un segnale
di disturbo è la sua ampiezza massima nel tempo, ossia il suo valore di \textbf{picco}.
L'inviluppo di un segnale infatti può essere variabile nel tempo.

Per questo motivo
il segnale ottenuto con il rilevatore di inviluppo viene inviato ad un filtro passa-basso
comunemente chiamato \textit{Video Pass Bandwidth}, viene quindi eseguita
la media dell'inviluppo. L'ampiezza di banda del filtro passa basso determina
il tempo necessario ad ottenere il valore di media.

Due segnali con lo stesso valore di picco possono però avere inviluppi differenti
in base alla frequenza con la quale il disturbo colpisce il rilevatore.

\begin{figure}[h] %2 circuiti con stesso picco ma periodicità differenti
 \centering
 \begin{subfigure}[t]{0.3\textwidth}
  \begin{circuitikz}
   \draw (0,0)[->] -- (4,0) node[right]{$t$};
   \draw (0,0)[->] -- (0,2) node[left]{$A(t)$};
   \draw (0.1,0) \foreach \k in {0,...,8} { let \n1={1.8*(-2*mod(\k,2) + 1)}
   in -- ++(0,\n1) -- ++(0.4,0)};
   \draw (3.7,0) -- (3.7,1.8);
  \end{circuitikz}
  \caption{}
  \label{fig:segnale_alta_periodicità}
 \end{subfigure}
 \quad \quad \quad
 \begin{subfigure}[t]{0.3\textwidth}
  \begin{circuitikz}
   \draw (0,0)[->] -- (4,0) node[right]{$t$};
   \draw (0,0)[->] -- (0,2) node[left]{$A(t)$};
   \draw (0.1,0) \foreach \k in {0,1} { let \n1={1.8*(-2*mod(\k,2) + 1)}
   in -- ++(0,\n1) -- ++(0.4,0)};
   
   \draw (1.7,0) \foreach \k in {0,1} { let \n1={1.8*(-2*mod(\k,2) + 1)}
   in -- ++(0,\n1) -- ++(0.4,0)};
  
  \draw (3.3,0) \foreach \k in {0} { let \n1={1.8*(-2*mod(\k,2) + 1)}
   in -- ++(0,\n1) -- ++(0.4,0)};
  \draw (3.7,0) -- (3.7,1.8);
  \end{circuitikz}
  \caption{}
  \label{fig:segnale_bassa_periodicità}
 \end{subfigure}
 \caption{}
\end{figure}

Come si vede il segnale in figura \ref{fig:segnale_alta_periodicità} ha un peso 
maggiore rispetto a quello in figura \ref{fig:segnale_bassa_periodicità} dato
che colpisce il ricevitore un numero maggiore di volte nella stessa unità
di tempo.
Il rivelatore di picco non è quindi in grado di discernere la differenza tra i due
segnali.

Si definisce quindi il valore di \textbf{Quasi Picco} quello ottenuto con l'aggiunta
di un condensatore in parallelo alla resistenza del rilevatore di picco, se ne riporta
al display del misuratore il suo valore medio.

La visualizzazione dei valori sul display viene pilotato mediante un generatore
di rampa che permette la traslazione della frequenza dell'oscillatore locale
e la traslazione sull'asse delle frequenze dello schermo.
