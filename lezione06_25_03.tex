
Un filtro con risposta in frequenza molto stretta, richiede una risposta nel tempo
molto ampia, ossia richiede molto tempo per andare a regime.

Fissata una RBW del filtro intermedio a \SI{3}{\decibel}, si avrà in uscita
un segnale composto dalla $f_{\text{if}}$ e da alcune componenti laterali, ossia
un segnale modulato in ampiezza la cui portante è proprio la $f_{\text{if}}$.
L'unione dei picchi nel tempo del segnale viene chiamato \textbf{inviluppo} del segnale,
è in realtà la parte ``interessante'' del segnale che viene infatti inviato
al \textit{rivelatore di inviluppo} che trasforma il segnale modulato nel suo inviluppo.

\begin{figure}[h]
\centering
 \begin{circuitikz}[american voltages]
 \draw
 (0,2) to [full diode,o-] (2,2)
       to [resistor] (2,0)
 (0,0) to [short, o-] (2,0)
 ;
 \draw
 (-0.5,2) to [open, v=$V_{\text{in}}$] (-0.5,0)
 (2.6,2) to [open, v^>=$V_{\text{out}}$] (2.6,0)
 ;
 \end{circuitikz}
 \caption{Rivelatore di inviluppo}
\end{figure}


