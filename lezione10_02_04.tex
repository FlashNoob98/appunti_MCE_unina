
\section{Segnali a banda stretta e banda larga}
Si definisce segnale a \textit{banda stretta} per antonomasia, un segnale sinusoidale
con una sola componente spettrale ma in generale
si definisce un segnale a banda stretta se il suo contenuto spettrale
sia interamente contenuto nella banda del filtro a frequenza intermedia.
Se il suo contenuto spettrale è al di fuori dell'ampiezza del filtro,
viene definito segnale a \textit{banda larga}.

Sia dato un segnale ad onda quadra così definito:
$$
x(t) = \sum_{k = -\infty}^{+\infty} A\cdot p\left(\frac{t-kT}{\tau}\right)
$$

La sua trasformata di Fourier è la seguente:
$$
x(f) = 2 \frac{A\tau}{T} \sum_{k=-\infty}^{+\infty} \text{sinc} \left(\pi \frac{k\tau}{T}\cdot  \delta\left(f-\frac{k}{T}\right)  \right)
$$

ossia un treno di impulsi con una distanza pari a $1/T$ e modulati
da una \textit{sinc} con impulsi che si azzerano in multipli di $1/\tau$.

Supponiamo di analizzare un segnale composto da impulsi ad onda quadra,
analizziamo l'intero spettro in un certo tempo $t$ definito come \textit{sweep time}
ossia il tempo necessario a passare dalla frequenza minima a quella massima,
il segnale però è presente soltanto durante la presenza dell'impulso.

\begin{figure}[h] %%grafico linee scansione sweep time
    \centering
    \def\svgwidth{0.6\columnwidth}
    \input{img/grafico.pdf_tex}
\end{figure}


Supponiamo di avere uno sweep time multiplo n-esimo
del tempo di ripetizione del segnale, rileveremo con una sola ``sweppata''
soltanto alcune frequenze, le frequenze analizzate durante il periodo di \textit{off}
del segnale, essa non verrà rilevata, l'esito dell'analisi sarà quindi povera di campioni,
la sua ricostruzione non restituirebbe il segnale originale.

Per ovviare a questo problema si può ridurre lo sweep time ed effettuare più scansioni
ripetute, conservando tutti i campioni raccolti si può (dopo un numero significativo di sweeppate)
ricostruire il segnale di partenza. Si può anche supporre che il tempo totale necessario ad effettuare
la misura sia lo stesso.
Si può inoltre affermare che il segnale sia di tipo ``impulsivo'' data l'assenza di campioni in
alcune misurazioni, determinato ciò, vanno determinati $T$ e $\tau$.
$T$ è la distanza tra 2 impulsi, utilizzando il grafico ottenuto con lo sweep lungo,
possiamo ricavare $T$ con la seguente:
$$
    \frac{\text{SWT}}{\text{SPAN}} = \frac{T}{\Delta f} \Rightarrow T = \frac{\text{SWT}\cdot\Delta f}{\text{SPAN}}
$$
%%% 1:00:30


