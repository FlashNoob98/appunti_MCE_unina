
\section{Camera semianecoica}
Nel caso in cui si voglia eseguire una prova di compatiblità elettromagnetica
in un ambiente non isolato, è necessario utilizzare un ambiente chiuso,
puro elettromagneticamente.
La camera chiusa viene resa perfettamente riflettente utilizzando una gabbia di Faraday esterna,
aumenterebbero però i punti di riflessione interni.

Per ridurre le riflessioni interne vengono quindi utilizzati i pannelli assorbenti,
il primo strato è composto da pannelli di ferrite, oltre le centinaia di \si{\mega\hertz} la ferrite
non è più sufficiente ad attenuare i raggi riflessi, vanno quindi usati dei pannelli piramidali a base quadrata 
contenenti polveri di carbonio, conduttive. In questo modo l'ampiezza del campo riflesso sarà attenuata 
ma verrà comunque convogliato verso l'interno della parete, riflettendo più volte tra i pannelli stessi, 
si può stimare un'attenuazione di diverse decine di \si{\decibel}.

Inoltre la particolare forma dei pannelli rende l'impedenza vista dal fronte d'onda gradualmente maggiore
riducendo al minimo la discontinuità che comporterebbe una forte riflessione.

Il piano di terra non viene ricoperto con materiale assorbente perchè deve garantire una perfetta
riflessione del campo, così come avveniva nel campo aperto.

(pag. 53 norma)
Date le dimensioni ridotte dell'ambiente, è possibile che alcuni punti dell'antenna ricevente si trovino a
distanze tali dalle pareti da risentire della distorsione del campo dovuto ai pannelli assorbenti.
Per caratterizzare l'ambiente di misura devo quindi eseguire almeno 20 misurazioni,
per caratterizzare un certo volume pari a quello occupato dall'antenna ricevente nell'ambiente,
si descriverà quindi un cilindro, composto da 4 misurazioni sulla circonferenza inferiore e superiore,
una al centro della circonferenza, per 2 altezze diverse e per entrambe le polarizzazioni,
per un totale appunto di 20 misurazioni di NSA differenti. Ogni singola misurazione deve rispettare 
le tolleranze dell'NSA teorico.

Conviene utilizzare sempre le antenne a banda larga per le misure in camera semianecoica
perché non si riuscirebbe a garantire una distanza minima di \SI{25}{\centi\meter} dalle pareti.
Le tabelle riportano il valore di NSA ad una distanza costante di \SI{3}{\meter}, è necessario
quindi spostare l'antenna ricevente sull'asse di mezzeria per compensare lo spostamento dell'antenna emittente.
Inoltre se si utilizzano antenne direttive, queste devono anche essere opportunamente ruotate
per puntare sempre in direzione dell'altra antenna, si deve sempre garantire il massimo accoppiamento.

Se l'antenna ricevente occupa una distanza su uno degli assi tale da coprire l'intero diametro o altezza
del cilindro, non è necessario ripetere le misure su quell'asse.

\newpage

\section{Misure di emissione radiata}
Una procedura di emissione radiata è pressocchè identica alla procedura di taratura,
nel primo caso si ha una sorgente nota, nel secondo l'EUT deve essere posto nelle condizioni peggiori
ossia nel punto di funzionamento in cui questo emetta più radiazioni, nel caso di un computer
ad esempio possono essere eseguiti dei software di benchmark, e il dispositivo deve essere posto nella sua
condizione di utilizzo tipica, ad esempio un tavolo o per terra.

Messo in funzione l'EUT deve ruotare e traslare in altezza, la risoluzione con la quale devono variare
questi parametri dipendono dalle dimensioni dell'antenna, %1:07


