
\paragraph{EMI receiver}

Questo dispositivo lavora con frequenze discrete, seleziona
una frequenza ed esegue la misura, poi l'oscillatore locale
avanza di una frequenza pari a $RBW/2$ ovvero mezza Resolution Bandwidth,
il tempo necessario ad eseguire la misura è molto maggiore a causa dei tempi
tipici dei vari rivelatori.

il \textit{Test della resolution bandwidth} permette di determinare
se il segnale è a banda stretta o larga, se il segnale è a banda stretta, il valore
misurato non dipende dall'ampiezza della Resolution Bandwidth (a meno del rumore addizionale),
nel caso di segnale a banda larga invece possiamo determinare un valore della Resolution Bandwidth oltre
il quale l'ampiezza non varia ulteriormente.
Questa procedura non è comunque esaustiva al fine della determinazione delle caratteristiche di 
ampiezza di banda del segnale.
%% inserisci figura 0:31

Un altro metodo è il confronto tra i valori dei rivelatori di picco e media,
il rivelatore di picco è sensibile a tutta la potenza del segnale presente nella
resolution bandwidth, mentre il rivelatore di average è sensibile prevalentemente al peso
del segnale alla frequenza centrale del filtro, se il valore di picco è superiore di \SI{6}{\decibel}
rispetto al valore di media, allora il segnale sarà a banda larga, se questa differenza
è inferiore a \SI{6}{\decibel} allora sarà a banda stretta.

\section{Disturbi radiati e disturbi condotti}
Esistono in Europa due range di frequenze per caratterizzare i disturbi radiati:
\begin{itemize}
 \item [FCC:] \SI{30}{\mega\hertz} ($\lambda = $ \SI{10}{\meter}) $\rightarrow$ \SI{40}{\giga\hertz} ($\lambda = $ \SI{7,5}{\milli\meter})
 \item [CISPR:] \SI{30}{\mega\hertz} ($\lambda = $ \SI{10}{\meter}) $\rightarrow$ \SI{1}{\giga\hertz} ($\lambda =$ \SI{30}{\centi\meter})
\end{itemize}

Entrambi gli enti normativi definiscono 2 categorie di oggetti da sottoporre a prove di compatibilità elettromagnetica,
\begin{itemize}
 \item [Classe A:] destinati all'industria pesante
 \item [Classe B:] destinati all'uso domestico e commerciale o industria leggera
\end{itemize}

\begin{figure}[h]
\begin{center}
\begin{tabular}{|c|c|c|}
  \hline
       & Classe A        & Classe B       \\ \hline
 FCC   & \SI{10}{\meter} & \SI{3}{\meter} \\ \hline
 CISPR & \SI{10}{\meter} &\SI{10}{\meter} \\ \hline
\end{tabular}
\end{center}
\caption{Distanza a cui effettuare le misure}
\end{figure}

Il diverso ambiente in cui questi oggetti devono funzionare rende necessaria la differente
analisi, gli oggetti di industria pesante sono solitamente di dimensioni maggiori,
questo rende necessario effettuare una misura a grande distanza per assicurarsi che 
ci si trovi in condizione di \textit{campo lontano}.

Nel caso in cui questa distanza non possa essere mantenuta, si effettua una misura
a distanza inferiore, ad esempio a \SI{3}{\meter} con valori di soglia superiori.

Sappiamo che il campo elettrico si propaga come:
$$
 \left|E_r \right| \propto \frac{1}{r} 
$$
il livello di campo a \SI{3}{\meter} si può calcolare:

$$
 \frac{E_3}{E_{10}} = \frac{r_{10}}{r_{3}} \Rightarrow 20\log_{10}E_3 - 20\log_{10}E_{10} = 20\log_{10}\left( \frac{10}{3}\right) \simeq \SI{10}{\decibel}
$$
$$
 E_3 \simeq 3\cdot E_{10}
$$

\section{Correnti di modo comune e correnti di modo differenziale}
Le correnti di \textbf{modo comune} si hanno solitamente quando sono presenti masse
a potenziali diverse oppure sono presenti asimmetrie o imperfezioni nelle schede elettroniche.
Le correnti di \textbf{modo differenziale} invece sono solitamente quelle necessarie al
corretto funzionamento di qualsiasi dispositivo elettrico o elettronico.

\begin{figure}[h] %1:15 aggiungi figura
\centering
 \begin{circuitikz}
 \draw
 (0,0) [dashed] -- (4,0)
 ;
 \end{circuitikz}
\end{figure}

Dalla figura si può vedere che il contributo di disturbo dato dalle
correnti di modo comune genera un campo nel punto 2 superiore a quello
generato dalle correnti di modo differenziale supponendo che la corrente nei conduttori sia la stessa.
Anche in caso di corrente di modo comune di intensità più bassa può comunque avere un'entità di disturbo
paragonabile a quello dato dalle correnti di modo differenziale.
