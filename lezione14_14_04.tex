
\section{Richiami di elettromagnetismo}
\paragraph{Legge di Ampere}
Immaginiamo un circuito chiuso all'interno del quale ci siano
sia correnti che entrano, che correnti che escono. %%%Inserisci immagine circuito

$$ %%legge di Ampere
\oint_C \vec{H} \cdot \vec{dl} = \int_S \vec{J}\cdot \vec{ds} + \frac{d}{dt}\left( \varepsilon \int_S \vec{E}\cdot\vec{ds}\right)
$$

La circuitazione del campo magnetico attraverso una superficie chiusa $S$ è dato dalla somma delle correnti
che attraversano quella superficie e dalla variazione di campo elettrico attraverso la stessa superficie (campo di
spostamento).

Utilizzando un ambiente schermato è possibile azzerare il contributo dovuto al campo elettrico,
isolando soltanto il contributo dovuto alle correnti nei conduttori, sfruttando questo principio
è possibile costruire una \textbf{sonda amperometrica} utilizzando un toro di ferrite a sezione rettangolare
sul quale viene posto un avvolgimento, ai suoi capi viene letta la tensione indotta.

$$
\Phi_B = \int_A \mu\vec{H} \cdot \vec{ds} \stackrel{N_{\text{spire}}}{\Rightarrow} V_{\text{emf}} = -N\frac{d}{dt} \mu \int_A \vec{H} \cdot \vec{ds} = -N\frac{d}{dt}\mu\Phi_A
$$

Si può quindi rappresentare la sonda come un doppio bipolo, il cui ingresso è la corrente da misurare
e la grandezza in uscita è la tensione ai suoi capi.
Si valuta quindi l'\textit{impedenza di trasferimento} indicata in \si{\decibel\ohm}
$$
\left|Z_T\right| = \frac{|V_{\text{out}}|}{|I_{\text{in}}|}
$$
Si vede che il valore massimo di impedenza si ottiene ad un valore di circa \SI{100}{\mega\hertz},
sfruttando la curva di taratura è possibile configurare lo strumento di misura correttamente per
riportare il corretto valore di corrente ad una data tensione misurata.
$$
|I|_{\si{\decibel\ampere}} = |V|_{\si{\decibel\volt}} - |Z_T|_{\si{\decibel\ohm}}
$$
Per eseguire la taratura viene \underline{ovviamente} utilizzato un carico di \SI{50}{\ohm}, 
la legge di Faraday, vale infatti per un circuito aperto, la tensione misurata sarà il risultato
della f.e.m. su un determinato carico.

Effettuando diverse misurazioni in un sistema monofase ad esempio si può misurare la presenza
di correnti di modo differenziale o di modo comune se si misurano i cavi di fase e neutro, escludendo 
dalla misura il conduttore di terra, o viceversa misurare solo il conduttore di terra per determinare
se le correnti di modo comune ritornano a terra mediante le masse del dispositivo o vengono radiate.

Se si misura un singolo cavo, il valore di campo elettrico è pari alla metà di quello precedentemente
calcolato:
$$
|E_C| = 2\pi\cdot10^{-7} \frac{fL}{d}|I_C| = 2\pi\cdot10^{-7}\frac{fL}{d}\frac{|V|}{|Z_T|}
$$

Sfruttando il valore di $Z_T$ si può esprimere il valore del campo in funzione del valore di tensione.
Per valutare le correnti di modo comune, andrebbe misurato il campo emesso dal dispositivo, in camera
semianecoica, utilizzando la 
relazione precedente si può eseguire la stessa misura utilizzando una sonda di corrente.
Si deve escludere preventivamente che le correnti di modo differenziale ad alta frequenza (non funzionali
al dispositivo) non ce ne siano.
Poichè in questo caso non vale il modello di dipolo Hertziano, la misura deve essere effettuata 
necessariamente a \SI{15}{\centi\meter} dalla porta di alimentazione del dispositivo per garantire
la riproducibilità della misura.

\section{Ambienti di misura}
A parità di sorgenti, l'ambiente in cui viene effettuata la misura ne influenza il risultato,
l'ambiente si comporta come un filtro, bisogna essere certi che a parità di disturbo i diversi 
ambienti influiscano allo stesso modo sulla misura.

\paragraph{Open Area Test Site}CISPR 16-1-4 Il miglior ambiente per eseguire una misura è l'\textit{OATS},
ossia un ambiente in campo aperto, libero da ostacoli, con un piano di terra privo di irregolarità.
L'EUT (Equipement Under Test) non può essere sospeso nel vuoto, va poggiato sul piano di utilizzo
tipico (come ad esempio un tavolo in legno alto \SI{80}{\centi\meter}).
Il piano di terra deve avere una conducibilità $\sigma$ tendente a $+\infty$, dato che la misura deve
essere eseguita a \SI{3}{\meter}, non mi interessa che il piano di misura si estenda all'infinito
bensì si delimita il perimetro del sito pari ad un ellisse i cui fuochi siano l'antenna di misura 
e l'EUT, il perimetro si estenda per una lunghezza pari a $2R$ in un senso e $\sqrt{3}R$ nell'altro.
Non devono essere presenti ostacoli all'interno del perimetro, la lunghezza totale minima che 
il segnale dovrebbe percorrere colpendo un ostacolo deve essere almeno pari a $2R$.

È importante che il piano di terra abbia una conducibilità tendente all'infinito affinché si possa supporre
che il campo irradiato dalla sorgente venga riflesso perfettamente.

Bisogna inoltre assicurarsi che non vi siano ulteriori apparecchi che emettono nella stessa banda, vengono
definiti 4 livelli di ``purezza'' elettromagnetica:
\begin{itemize}
\item Tutti i disturbi sono almeno di \SI{6}{\decibel} al di sotto dei valori misurati con dispositivo spento
\item Alcune emissioni nell'ambiente al di sotto di meno di \SI{6}{\decibel} dei valori misurati
\item Emissioni discontinue ma al di sopra di \SI{6}{\decibel} dei valori misurati
\item Emissioni al di sopra dei livelli per una porzione significativa della frequenza di misura e che 
avvengono con continuità.
\end{itemize}
