
\section{Immunità Radiata}
L'analisi di immunità radiata copre le frequenze da \SI{20}{\mega\hertz} a \SI{3}{\giga\hertz}.
Si devono definire i parametri di immunità di un dispositivo, vanno valutate le degradazioni di prestazioni
del singolo dispositivo.

Esistono quattro criteri di accettazione:
\begin{itemize}
 \item [A] Il dispositivo continua a comportarsi come prima della attivazione del disturbo
 \item [B] Il dispositivo presenta una degradazione delle prestazioni che termina al terminare del disturbo 
 \item [C] Degradazione delle prestazioni che richiede un intervento manuale o automatico per il loro ripristino
 \item [D] Criterio di non accettazione, il dispositivo presenta delle degradazioni non recuperabili
\end{itemize}

Assicurarsi una buona immunità dai disturbi elettromagnetici, oltre a rispettare le norme si ottiene
anche un prodotto qualitativamente migliore.

Per eseguire la prova si deve tarare un certo campo di disturbo, va tarato il disturbo ad un valore
ad esempio di \SI[per-mode=symbol]{3}{\volt\per\meter} uniforme su un piano di \SI{1.5}{\meter} per \SI{1.5}{\meter}.

È difficile avere in queste condizioni un campo uniforme data la piccola distanza della superficie dall'emettitore,
al campo diretto bisogna aggiungere quello riflesso sul pavimento, per migliorare la situazione riguardo 
l'uniformità di campo si ottiene anecoicizzando il pavimento.
In ogni caso il fronte d'onda non potrà mai essere del tutto uniforme, si assume quindi un campo uniforme con una 
tolleranza di \SI{+6}{\decibel}. Si assume una tolleranza solo positiva perchè il campo dichiarato 
deve essere il minimo al quale il dispositivo deve essere dichiarato immune.
Il valore del campo $E_T$ si può esprimere come
$$
E_T = V_G + A - \alpha + AF_T -A_d
$$
Dove $V_G$ è la tensione del generatore, $A$ il guadagno dell'amplificatore, $\alpha$ l'attenuazione
del cavo, $AF_T$ il fattore d'antenna e $A_d$ l'attenuazione del campo dovuta alla distanza.
Il range di frequenze utilizzate varia da un minimo di \SI{20}{\mega\hertz} a \SI{3}{\giga\hertz},
questo implica che tutti i parametri presenti nella formula variano con la frequenza, sarebbe molto difficile
avere un modello accurato, piuttosto si preferisce agire diversamente, si misura il campo
sul piano e si regola la tensione del generatore affinchè il valore di campo non sia 
pari a quello desiderato.

Per effettuare la taratura sul punto di misura si usa una sonda e si campionano dei punti presenti nel
piano, la norma prevede la suddivisione dello spazio in 16 punti, si esegue la
misura su un punto e si considera come punto di taratura, ossia si memorizzano
i valori di potenza per ogni frequenza e si usano gli stessi valori sugli altri
punti da caratterizzare.

Si otterrà un insieme di 16 curve differenti campo/frequenza, si calcola a questo punto
il campo medio $\overline{E}$ e si tracciano 2 linee pari a $\overline{E} + \SI{3}{\decibel} $ ed $\overline{E} - \SI{3}{\decibel} $.
Se questa \textit{fascia} di \SI{6}{\decibel} attorno al valore medio
contiene tutte le 16 curve, si sarà già garantita l'uniformità di campo.
Supponiamo che ci siano alcuni punti che per qualche frequenza escano fuori dalla fascia
appena indicata, fino ad un massimo del 25\% dei punti totali, si può ritenere il campo
uniforme.

Se il numero di punti al di fuori della fascia è superiore al 25\% non si può garantire l'uniformità di campo. Si può cercare di compensare il campo spostando l'antenna.

%1:13
Ottenuto un capo uniforme, va garantito che il campo minimo diventi il campo $E_T$,  
si può pensare quindi di attuare una variazione di potenza uniforme per tutti i punti
di misura, per ogni frequenza, in modo da portare il campo minimo al campo target.
$$
\Delta E = 20\log_{10}\frac{E_T}{E_{\text{min}}} = 10\log_{10}\frac{P_T}{P_{\text{min}}} = \Delta P
$$

In questo modo l'uniformità sarà data dalla distanza fra il punto massimo e minimo
tra tutte le curve disponibili. Per migliorare l'uniformità si può in alternativa
calcolare una variazione di potenza per ogni valore di frequenza, in questo modo
per ogni frequenza, ci sarà almeno uno dei 16 punti con un valore di campo pari ad $E_T$.
In questo modo l'uniformità di campo sarà minore dell'uniformità di campo ottenuta con il
secondo metodo.
%1:36

Durante la fase di taratura si utilizza un segnale sinusoidale e si determinano le potenze.
Il segnale da inviare al dispositivo durante la prova sarà un segnale sinusoidale modulato in ampiezza
con una profondità di modulazione all'80\% affinchè l'EUT non si ``abitui'' al disturbo ricevuto.

Un tipo di approccio diverso alla taratura è quello con potenza costante (e non campo costante),
si sceglie per il punto 1 una potenza target costante pur tenendo un campo di almeno
\SI[per-mode=symbol]{3}{\volt\per\meter} affinchè sia correttamente misurabile.
