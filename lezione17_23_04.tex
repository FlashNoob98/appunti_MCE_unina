
\section{Immunità Radiata}
L'analisi di immunità radiata copre le frequenze da \SI{20}{\mega\hertz} a \SI{3}{\giga\hertz}.
Si devono definire i parametri di immunità di un dispositivo, vanno valutate le degradazioni di prestazioni
del singolo dispositivo.

Esistono quattro criteri di accettazione:
\begin{itemize}
 \item [A] Il dispositivo continua a comportarsi come prima della attivazione del disturbo
 \item [B] Il dispositivo presenta una degradazione delle prestazioni che termina al terminare del disturbo 
 \item [C] Degradazione delle prestazioni che richiede un intervento manuale o automatico per il loro ripristino
 \item [D] Criterio di non accettazione, il dispositivo presenta delle degradazioni non recuperabili
\end{itemize}

Assicurarsi una buona immunità dai disturbi elettromagnetici, oltre a rispettare le norme si ottiene
anche un prodotto qualitativamente migliore.

Per eseguire la prova si deve tarare un certo campo di disturbo, va tarato il disturbo ad un valore
ad esempio di \SI[per-mode=symbol]{3}{\volt\per\meter} uniforme su un piano di \SI{1.5}{\meter} per \SI{1.5}{\meter}.

È difficile avere in queste condizioni un campo uniforme data la piccola distanza della superficie dall'emettitore,
al campo diretto bisogna aggiungere quello riflesso sul pavimento, per migliorare la situazione riguardo 
l'uniformità di campo si ottiene anecoicizzando il pavimento.
In ogni caso il fronte d'onda non potrà mai essere del tutto uniforme, si assume quindi un campo uniforme con una 
tolleranza di \SI{+6}{\decibel}. Si assume una tolleranza solo positiva perchè il campo dichiarato 
deve essere il minimo al quale il dispositivo deve essere dichiarato immune.
Il valore del campo $E_T$ si può esprimere come
$$
E_T = V_G + A - \alpha + AF_T -A_d
$$
Dove $V_G$ è la tensione del generatore, $A$ il guadagno dell'amplificatore, $\alpha$ l'attenuazione
del cavo, $AF_T$ il fattore d'antenna e $A_d$ l'attenuazione del campo dovuta alla distanza.
Il range di frequenze utilizzate varia da un minimo di \SI{20}{\mega\hertz} a \SI{3}{\giga\hertz},
questo implica che tutti i parametri presenti nella formula variano con la frequenza, sarebbe molto difficile
avere un modello accurato, piuttosto si preferisce agire diversamente, si misura il campo
sul piano e si regola la tensione del generatore affinchè il valore di campo non sia 
pari a quello desiderato.

Per effettuare la taratura sul punto di misura si usa una sonda e si campionano dei punti presenti nel
piano, la norma prevede la suddivisione dello spazio in 16 punti

