\section{Enti normativi}
Il \textbf{CENELEC} è il comitato europeo di normazione elettrotecnica, in Italia invece l'ente normativo
è il \textbf{CEI}, Comitato Elettrotecnico Italiano che recepisce le normative europee e in genere le recepisce
traducendole senza apportare alcuna modifica, se il CEI ha invece già emanato una norma sull'argomento deve 
subito provvedere ad aggiornarla per renderla conforme con la normativa CENELEC, solitamente viene recepita
direttamente la norma CENELEC, ritirando quella italiana.

\textbf{Tipi di norme:}
\begin{itemize}
 \item \textbf{Base}: prettamente metodologica, descrizione della metodologia di prova, della strumentazione di misura
 con le sue caratteristiche, calibrazione dello stesso e ulteriori prescrizioni sui metodi di misura
 durante la validazione dell'elemento in prova. Non fissa alcun limite.
 \item \textbf{Generiche}: forniscono dei limiti e differenziano gli ambienti in cui i dispositivi vengono 
 utilizzati, ad esempio la suddivisione tra ambiente domestico, dove i dispositivi possono essere molto vicini
 tra loro, e l'ambiente industriale dove i dispositivi sono posti a distanze ragionevoli ed inoltre l'industria
 o l'azienda possiedono i fondi necessari alla ricerca di eventuali problemi di compatibilità.
 \item \textbf{Di prodotto}: fissano anch'esse dei limiti ma riguardano singoli prodotti o categorie di prodotti.
 Se per un prodotto non esiste una specifica norma, si applica la norma generica.
 \item \textbf{Armonizzate}: sono norme generiche o di prodotto, fatte proprie dall'unione europea e recepite,
 ``armonizzate'' dai vari stati includendole nel loro corpus normativo.
 \end{itemize}

Non sempre è necessario eseguire prove normate, ci si può affidare ad organismi terzi per verificare
il soddisfacimento dei requisiti. È possibile inoltre dimostrare la compatibilità elettromagnetica del proprio
prodotto utilizzando unicamente il progetto, se questo è in grado di dimostrare intrinsecamente il 
soddisfacimento dei requisiti imposti. È comunque preferito molto spesso l'approccio sperimentale per
la difficoltà molto spesso di avere un modello che copra tutti i range di frequenze.

Il CENELEC fu fondato nel '73 e composto dai comitati tecnici dei singoli paesi europei, inclusi affiliati
esterni che partecipano alle discussioni ma senza diritto di voto, è finalizzato all'armonizzazione: raccoglie 
ed elabora le norme emesse da altri enti (es. IEC) al fine di garantire uno standard richiesto dal mercato
europeo.

L'origine di una norma si riconosce mediante la sua sigla, ad esempio EN 50157-2-1.
Le normative europee sono così numerate:
\begin{itemize}
 \item \textbf{40000/44999} derivano da una standardizzazione congiunta del CEN e del CENELEC riguardo
 il settore IT.
 \item \textbf{45000/49999} riguardano le attività congiunte CEN e CENELEC al di fuori del settore IT.
 \item \textbf{50000/59999} riguardano le attività esclusive del CENELEC.
 \item \textbf{60000/69999} l'implementazione da parte del CENELEC delle norme IEC.
\end{itemize}
Ad esempio la normativa europea EN 61000-4-3 deriva dalla IEC 1000-4-3 con le eventuali modifiche, la 
norma CISPR-16 è stata recepita in Europa con il numero EN 55016. %00:33



