
\paragraph{Calcolo campo con correnti di modo comune}
$I_1 = I_2 = I_C$

\begin{equation*}
\begin{split}
E_C & = 2 \pi j \cdot 10^{-7} L f \frac{I_C}{r}e^{-j \beta_0 r} \left[e^{-j \beta_0 \Delta} + e^{j \beta_0 \Delta} \right]\sin\theta \\
\left|E_C\right| & = 2 \pi \cdot 10^{-7} L f \frac{I_C}{r} \cdot 2 \cos(\beta_0\Delta)\sin\theta = \\
& = 4\pi\cdot10^{-7}Lf \frac{I_C}{r}\cos\left.\left(\frac{2\pi}{\lambda}\frac{s}{2}\sin\theta\sin\varphi\right)\sin\theta\right|_{\theta=\frac{\pi}{2}} =\\
& = 4\pi\cdot10^{-7}Lf\frac{I_C}{r}\cos\left.\left(\frac{2\pi}{\lambda}\frac{s}{2} \sin\varphi\ \right)\ \right|_{\varphi=?} \stackrel{s\ll\lambda}{=} 4\pi\cdot10^{-7}Lf\frac{I_C}{r}
\end{split}
\end{equation*}

Si riportano quindi le due espressioni di campo elettrico in funzione delle correnti:

\begin{equation*}
 \begin{split}
 E_D & = \frac{4\pi^2\cdot10^{-7}}{c}\cdot \frac{I_D}{r} f^2 Ls\ \text{con}\ \begin{split}
 \theta & = \pi/2 \\
 \varphi & = \pi/2
 \end{split}\\
 E_C & = 4\pi\cdot10^{-7}\frac{I_C}{r}fL\qquad\ \ \text{con}\ \begin{split}
  \theta & = \pi/2 \\
 \forall& \ \varphi\in R
 \end{split}
 \end{split}
\end{equation*}

\paragraph{Esercizio}
Siano dati i seguenti valori: (Sol. $I_C = \SI{2.4e-6}{\ampere}\quad I_D = \SI{1.8e-3}{\ampere}$)
\begin{equation*}
 \begin{split}
 f & = \SI{100}{\mega\hertz} \qquad L = \SI{1}{\meter} \\
 d & = \SI{3}{\meter}\quad \ \ s = 50\ \text{mil} = 50\cdot 25,4\cdot10^{-6} \si{\meter}
 \end{split}
\end{equation*}
Trova i valori di corrente $I_C$ e $I_D$ tali che si abbia un campo pari a $E = \SI[per-mode=symbol]{100}{\micro\volt\per\meter}$

\newpage

%%inizia pagina nuova
\paragraph{Contromisure}
Lo scopo dell'analisi di compatibilità elettromagnetica è quello di attuare opportune
contromisure per ridurre l'intensità del campo, alcuni parametri delle equazioni sono costanti,
si può quindi agire sulla corrente necessaria al funzionamento del dispositivo
riducendo al minimo $I_D$ affinché funzioni il dispositivo, le correnti $I_C$ invece sono correnti 
dovute a non idealità che vanno quindi ridotte al minimo, ad esempio gestendo al meglio
filtraggi e bilanciamenti, la presenza di masse a potenziale differente.

Un ulteriore soluzione è la riduzione della frequenza del segnale, in particolare
per il campo di modo differenziale che dipende dal quadrato della frequenza,
la controindicazione è avere un dispositivo che trasporta meno informazioni.

I fattori geometrici come la distanza tra i conduttori e la loro lunghezza influiscono
anch'essi sulla ampiezza del campo emesso, all'aumentare di tali parametri aumenta il campo emesso,
per le correnti di modo comune conta l'\textit{area} mentre per quelle di modo differenziale la \textit{lunghezza}.

Definiamo $E/I$ come la funzione di trasferimento del sistema corrente-campo, ingresso-uscita:
\begin{equation*}
\begin{split}
\frac{E_D}{I_D} & = K(d)f^2 \\
\frac{E_C}{I_C} & = K'(d)f
\end{split}
\end{equation*}
Rappresentiamo il diagramma di Bode della funzione di trasferimento per il campo differenziale:

%%Inserisci diagramma 0:57

Abbiamo una funzione che sale con una pendenza di \SI{+40}{\decibel/dec} mentre nel caso della funzione
di modo comune avremo una pendenza di \SI{+20}{\decibel/dec}, il segnale in ingresso
è un treno di impulsi con parametri $\tau$ il periodo e $\tau_r$ il tempo di salita.
Il diagramma di Bode del segnale parte con una pendenza di \SI{0}{\decibel/dec} e alla frequenza di $1/\pi\tau$
scende con una pendenza di \SI{-20}{\decibel/dec}, in corrispondenza di $1/\pi\tau_r$ continua
a scendere con pendenza \SI{-40}{\decibel/dec}.

Sommando i vari diagrammi si può ottenere il campo in uscita data la corrente.
%%Inserisci diagrammi risultati
Osservando i grafici si può vedere l'andamento di un pacchetto di frequenze che formano il treno %1:08 
di impulsi trapezoidali si vede che il segnale di modo
differenziale raggiunge il picco dopo una certa frequenza, per ridurre l'emissione si può aumentare
$\tau$ e $\tau_r$ ottenendo complessivamente una pendenza più bassa e un valore assoluto del segnale
inferiore, allo stesso modo anche il segnale di modo comune, non è però conveniente ridurre le frequenze dei funzionamento dei dispositivi.
Una soluzione utile per ridurre i disturbi emessi può essere quella di intrecciare i cavi di alimentazione

