
\paragraph{Calcolo campo con correnti di modo comune}
$I_1 = I_2 = I_C$

\begin{equation*}
\begin{split}
E_C & = 2 \pi j \cdot 10^{-7} L f \frac{I_C}{r}e^{-j \beta_0 r} \left[e^{-j \beta_0 \Delta} + e^{j \beta_0 \Delta} \right]\sin\theta \\
\left|E_C\right| & = 2 \pi \cdot 10^{-7} L f \frac{I_C}{r} \cdot 2 \cos(\beta_0\Delta)\sin\theta = \\
& = 4\pi\cdot10^{-7}Lf \frac{I_C}{r}\cos\left.\left(\frac{2\pi}{\lambda}\frac{s}{2}\sin\theta\sin\varphi\right)\sin\theta\right|_{\theta=\frac{\pi}{2}} =\\
& = 4\pi\cdot10^{-7}Lf\frac{I_C}{r}\cos\left.\left(\frac{2\pi}{\lambda}\frac{s}{2} \sin\varphi\ \right)\ \right|_{\varphi=?} \stackrel{s\ll\lambda}{=} 4\pi\cdot10^{-7}Lf\frac{I_C}{r}
\end{split}
\end{equation*}

Si riportano quindi le due espressioni di campo elettrico in funzione delle correnti:

\begin{equation*}
 \begin{split}
 E_D & = \frac{4\pi^2\cdot10^{-7}}{c}\cdot \frac{I_D}{r} f^2 Ls\ \text{con}\ \begin{split}
 \theta & = \pi/2 \\
 \varphi & = \pi/2
 \end{split}\\
 E_C & = 4\pi\cdot10^{-7}\frac{I_C}{r}fL\qquad\ \ \text{con}\ \begin{split}
  \theta & = \pi/2 \\
 \forall& \ \varphi\in R
 \end{split}
 \end{split}
\end{equation*}

