%\documentclass[a4paper,11pt]{article}
\documentclass[a4paper,11pt]{scrartcl}

\usepackage[utf8]{inputenc}

\title{Appunti di compatibilità elettromagnetica}
\author{Daniele Olivieri}
\date{}

\pdfinfo{%
  /Title    ()
  /Author   ()
  /Creator  ()
  /Producer ()
  /Subject  ()
  /Keywords ()
}

\begin{document}
\maketitle

\section{Cenni sulla compatibilità elettromagnetica}

Cos'è la compatibilità elettromagnetica? La scienza che studia la capacità di qualsiasi 
apparecchiatura di funzionare in modo soddisfacente in un ambiente soggetto a disturbi elettromagnetici
senza produrre disturbi intollerabili ad altri apparati.

Ad esempio quando si avvicina il cellulare alle casse dello stereo, si percepisce un ronzio nella cassa
dello stesso, questo fenomeno è dovuto al trasferimento di parte dell'energia emessa dal cellulare ai cavi 
dello stereo che lo convertono in disturbo sonoro.
Questo fenomeno non può e non deve avvenire nemmeno in situazioni critiche come una sala ospedaliera in cui il disturbo
emesso da un cellulare potrebbe alterare le misurazioni effettuate da un apparecchio come l'elettrocardiografo.
Il dispositivo elettromedicale deve essere reso immune dai disturbi. (Immagina un pacemaker!)

Si distingue la Compatibilità in senso attivo o passivo:
\begin{itemize}
 \item Attivo: l'oggetto non deve emettere disturbi di entità troppo elevata
 \item Passivo: l'oggetto deve essere in grado di resistere ai disturbi esterni, emessi da altri dispositivi
 senza alterare il proprio funzionamento
\end{itemize}

Altri dispositivi che devono sottoporsi alle analisi di Compatibilità sono i dispositivi comunque complessi
composti ad esempio da parti multiple, che possono essere commercializzate separatamente. In questo caso solo
le singole parti devono garantire un soddisfacimento dei vincoli dati dalla Compatibilità Elettromagnetica.

Si parla di Compatibilità \textbf{intra-sistema} se si analizza il problema all'interno del sistema con approccio
``microscopico'', la compatibilità \textbf{inter-sistema} analizza invece la compatibilità tra dispositivi differenti.
Esistono due grandi fenomeni: \textbf{emissione} e \textbf{immunità}, questi due fenomeni richiamano la compatibilità in senso attivo e passivo.

I disturbi vengono poi caratterizzati in disturbi \textbf{radiati} in aria e disturbi \textbf{condotti} tramite
un canale vincolato, ad esempio un cavo di alimentazione.
Nessun disturbo si può caratterizzare mediante una singola tipologia, %rivedi qui
un disturbo radiato ad esempio può accoppiarsi con il cavo di alimentazione e diventare disturbo condotto.
Il cilindro di ferrite presente su un cavo VGA ad esempio attenua i disturbi che tendono a propagarsi lungo il cavo
incrementandone l'impedenza.
\newpage
Al di sotto dei 30 MHz si ritiene che il fenomeno sia di natura prevalentemente condotta,
al di sopra invece si ritiene il fenomeno di natura radiata. Ad esempio il ``surge'' nasce da un fulmine e si
propaga per via radiata, ma raggiunge i dispositivi mediante i loro cavi di alimentazione.

\end{document}
