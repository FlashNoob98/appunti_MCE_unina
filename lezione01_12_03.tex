\section{Cenni sulla compatibilità elettromagnetica}
\paragraph{Definizione}
La compatibilità elettromagnetica è la scienza che studia la capacità di qualsiasi 
apparecchiatura di funzionare in modo soddisfacente in un ambiente soggetto a disturbi elettromagnetici
senza produrre disturbi intollerabili ad altri apparati.

Ad esempio quando si avvicina il cellulare alle casse dello stereo, si percepisce un ronzio nella cassa
dello stesso, questo fenomeno è dovuto al trasferimento di parte dell'energia emessa dal cellulare ai cavi 
dello stereo che lo convertono in disturbo sonoro.
Questo fenomeno non può e non deve avvenire nemmeno in situazioni critiche come una sala ospedaliera in cui il disturbo
emesso da un cellulare potrebbe alterare le misurazioni effettuate da un apparecchio come l'elettrocardiografo.
Il dispositivo elettromedicale deve essere reso immune dai disturbi. (Immagina un pacemaker!)

Si distingue la Compatibilità in senso attivo o passivo:
\begin{itemize}
 \item Attivo: l'oggetto non deve emettere disturbi di entità troppo elevata
 \item Passivo: l'oggetto deve essere in grado di resistere ai disturbi esterni, emessi da altri dispositivi
 senza alterare il proprio funzionamento
\end{itemize}

Altri dispositivi che devono sottoporsi alle analisi di Compatibilità sono i dispositivi comunque complessi
composti ad esempio da parti multiple, che possono essere commercializzate separatamente. In questo caso solo
le singole parti devono garantire un soddisfacimento dei vincoli dati dalla Compatibilità Elettromagnetica.

Si parla di Compatibilità \textbf{intra-sistema} se si analizza il problema all'interno del sistema con approccio
``microscopico'', la compatibilità \textbf{inter-sistema} analizza invece la compatibilità tra dispositivi differenti.
Esistono due grandi fenomeni: \textbf{emissione} e \textbf{immunità}, questi due fenomeni richiamano la compatibilità in senso attivo e passivo.
\medskip

I disturbi vengono poi caratterizzati in disturbi \textbf{radiati} in aria e disturbi \textbf{condotti} tramite
un canale vincolato, ad esempio un cavo di alimentazione.
Nessun disturbo si può caratterizzare mediante una singola tipologia, %rivedi qui
un disturbo radiato ad esempio può accoppiarsi con il cavo di alimentazione e diventare disturbo condotto.
Il cilindro di ferrite presente su un cavo VGA ad esempio attenua i disturbi che tendono a propagarsi lungo il cavo
incrementandone l'impedenza.
\newpage
Al di sotto dei 30 MHz si ritiene che il fenomeno sia di natura prevalentemente condotta,
al di sopra invece si ritiene il fenomeno di natura radiata. Ad esempio il ``\textbf{surge}'' nasce da un fulmine e si
propaga per via radiata, ma raggiunge i dispositivi mediante i loro cavi di alimentazione.

I problemi di compatibilità intra-sistema vengono gestiti durante la fase di progettazione del dispositivo
ad esempio mediante una corretta disposizione delle piste del circuito, attraverso una opportuna scelta dei 
componenti oppure mediante una schermatura interna, non disporrò mai un dispositivo molto suscettibile
in prossimità di un potenziale emettitore di disturbi sulla stessa scheda elettronica.

La compatibilità inter-sistema viene studiata mediante una caratterizzazione dell'ambiente elettromagnetico, 
mediante una classificazione delle sorgenti dei disturbi e delle vittime, attraverso la definizione dei livelli di massima
emissione e minima immunità, non riguarda soltanto l'oggetto ma anche la sua posizione nell'ambiente
elettromagnetico.
\medskip

L'\textbf{ambiente elettromagnetico} è l'insieme di tutti i fenomeni elettromagnetici osservabili in un
determinato luogo, sostanzialmente il \textbf{campo} elettromagnetico, le condizioni al contorno e le caratteristiche
del mezzo in cui i disturbi si propagano. Per modellare l'ambiente e ottenere un risultato valido
è necessaria una esatta conoscenza di tutti questi parametri, impresa per nulla semplice.

Anziché calcolare la soluzione numericamente, si preferisce infatti misurare direttamente l'ambiente
elettromagnetico e le emissioni generate dall'oggetto. Anche quest'aspetto presenta dei problemi come
l'incertezza di misura e la discretizzazione del fenomeno, ovvero la scelta del numero di punti in cui effettuare
la misura.
Si può scegliere di eseguire entrambe le procedure, utilizzare un modello numerico approssimato per 
prevedere a grandi linee i risultati e approfondire con le misurazioni i punti più critici.

\paragraph{La perturbazione elettromagnetica}
È un fenomeno di origine elettromagnetico capace di alterare il funzionamento di un'apparecchiatura,
può essere costituito da:
\begin{itemize}
\item Rumore
\item Segnale non desiderato
\item Alterazione del mezzo
\end{itemize}
