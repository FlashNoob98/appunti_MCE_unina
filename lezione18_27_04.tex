\section{Misura dei disturbi condotti}
Le misure effettuate riguardano i cavi di alimentazione dei dispositivi,
le normative vigenti impongono limiti di tensione indotta, ma la caratteristica
principale che emette i disturbi elettromagnetici è la corrente,
le frequenze di interesse partono da un valore di \SI{150}{\kilo\hertz} per le normative 
CISPR e \SI{450}{\kilo\hertz} per quelle FCC, fino ad una frequenza di 
\SI{30}{\mega\hertz}.
Sembra paradossale riferirsi a disturbi di campo radiato per frequenze relativamente basse,
va quindi ricordato che i disturbi condotti si propagano sulla rete elettrica, ramificata 
ma soprattutto molto estesa, per questo motivo anche frequenze basse possono irradiare
dei disturbi a distanze considerevoli.

\begin{figure}[h]
 \centering
 \begin{circuitikz}
 \draw (0,0) to [resistor, o-,l=$R_s$] (2,0)
            to [vsourcesin,l=$V_s$] (2,-2)
            to [short,-o] (0,-2) 
 ;
 \draw (0,-2) to [open,v=$V_n$] (0,0);
 \draw (-0.05,0) to [short,i=$ $] (-1,0)
        to [generic,l_=$Z_N$] (-1,-2)
        to [shor] (-0.05,-2);
 \end{circuitikz}
 \caption{Modello di EUT connesso alla rete elettrica}
\end{figure}

In prima ipotesi è possibile misurare la tensione di disturbo mediante un
voltmetro collegato all'uscita del dispositivo, la tensione di rumore $V_N$ sarebbe 
però pari a:
$$
V_N = V_s\frac{Z_N}{R_s+Z_N}
$$
ossia dipenderebbe dall'impedenza della rete, la misura non sarebbe ripetibile e 
nemmeno riproducibile.
Si deve quindi utilizzare una configurazione che isoli l'EUT dalla rete.
Si utilizza un dispositivo chiamato \textbf{LISN} (\textit{Line Impiedance Stabilization
Network}) interposto tra l'EUT e la rete elettrica.

\paragraph{LISN monofase}
La LISN in figura è della tipologia FCC.
%induttanza 50 \micro\henry
% C lato alimentazione 1\micro\farad
%C lato EUT 0.1 microfoarad
%resistenze: 1 kohm 50 ohm
%0:32
La LISN non deve offrire impedenza alla frequenza di \SI{50/60}{\hertz}, necessaria
all'alimentazione del dispositivo, le induttanze-serie si comportano come un
corto circuito. Un tipo di LISN utile a misurare i disturbi tra fase e terra
e neutro e terra, viene detta LISN a ``Y''. Esistono LISN a $\Delta$ utili a misurare
i disturbi tra le fasi.
I condensatori in parallelo invece si comportano come circuiti aperti per basse frequenze.

In alta frequenza invece (\SI{e5}{\hertz}) gli induttori serie si comportano come un
circuito aperto mentre i condensatori come corto circuiti.
$$
Z_L = 2\pi f L \qquad Z_C = \frac{1}{2\pi f C}
$$

Le coppie di resistenze in parallelo rappresentano rispettivamente
la resistenza d'ingresso del ricevitore e una resistenza da
\SI{1}{\kilo\ohm} necessaria ad abbattere la tensione sulla resistenza da \SI{50}{\ohm}
a bassa frequenza dovuta alla presenza del condensatore. Durante il cambio dei 
collegamenti dei morsetti di misura infatti, non si avrebbe controllo sulla tensione
flottante sul condensatore.
Le norme infatti impongono di misurare una volta la tensione tra fase e terra, 
e tra neutro e terra.
%0:50:46
Il condensatore lato EUT è fondamentale affinchè non vi siano \SI{230}{\volt} sull'EMI 
receiver.

L'impedenza vista dall'EUT, utilizzando la LISN, è sempre pari a \SI{50}{\ohm} per
frequenze maggiori del \si{\mega\hertz}. Viene prevista una fascia di tolleranza
del 20\% attorno al valore nominale di impedenza.
Questa tolleranza così elevata potrebbe però causare una notevole differenza tra
i risultati ottenuti con LISN differenti.

\paragraph{Filtro a $\pi$}Si riducono i disturbi emessi dall'EUT utilizzando un 
filtro a $\pi$. %1:25:30 inserisci figura

\begin{figure}[h] %Filtro a pi greco
\centering
 \begin{circuitikz}
 \draw (0,2) to [inductor,*-*] (10,2) node[right]{P};
 \draw (0,1) to [inductor,*-*] (10,1) node[right]{N};
 \draw (1,0) to [short,-*]    (10,0) node[right]{PE};
 \draw (0,0) to [inductor,*-,l_=$L_{\text{GW}}$]  (1,0);
 
 \draw (1,2) to [capacitor,*-,l=$C_c$] (1,1)
 to [short,-*] (1,0);
  \draw (9,2) to [capacitor,*-,l=$C_c$] (9,1)
 to [short,-*] (9,0);
 
 \draw (2,1) to [capacitor,*-*,l=$C_c$] (2,0);
  \draw (8,1) to [capacitor,*-*,l=$C_c$] (8,0);
  
  \draw (3,2) to [capacitor,*-*,l=$C_d$] (3,1);
\draw (7,2) to [capacitor,*-*,l=$C_d$] (7,1);
 \end{circuitikz}
 \caption{Esempio di filtro a $\pi$}
 \label{fig:filtro_pi}
\end{figure}

Il filtro rappresentato in figura \ref{fig:filtro_pi} è interno all'EUT.
I valori tipici sono:
\begin{itemize}
 \item [$C_d$] = \SI{47}{\nano\farad} devono resistere alla tensione di fase.
 \item [$C_c$] $\leq \frac{I_{\text{Lmax}}/2}{2\pi f \cdot V_{P/N}} \simeq 
 \SI{2200}{\pico\farad}$  con $I_{\text{Lmax}}$ la massima corrente di 
 leakage, si deve impedire che la corrente 
 nel cavo di terra, superi i limiti dell'interruttore differenziale.
\end{itemize}

È fondamentale integrare il filtro durante la progettazione della scheda, evitando
costi inutili aggiuntivi per trovare dispositivi discreti da integrare durante la fase
finale della realizzazione del dispositivo.
