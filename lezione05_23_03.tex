\section{Sorgenti e ricevitori di segnale}
È possibile rappresentare con dei circuiti equivalenti le sorgenti dei disturbi
e il ricevitore di interferenza 
\begin{figure}[h]
\centering
\begin{subfigure}[t]{0.4\textwidth}
\begin{circuitikz}
\draw
(0,0) to [sinusoidal voltage source, l=$V_s$] (0,2)
        to [resistor,-o,l=\mbox{$R_s$}, a=\SI{50}{\ohm}] (3.5,2)
(0,0) to [short , -o] (3.5,0)
;
\draw [dashed] (-0.5,-0.3) rectangle (3,2.25)
;
\end{circuitikz}
\caption{Generatore di disturbi}
\end{subfigure}
\ 
\begin{subfigure}[t]{0.4\textwidth}
\begin{circuitikz}
\draw
(0,0) to [open, o-o] (0,2)
        to [short] (3.5,2)
        to [resistor , l=\parbox{2cm}{$R_{in}\\ \SI{50}{\ohm}$}] (3.5,0)
        to [short] (0,0)
(2.5,0)   to [capacitor, l=\parbox{2.3cm}{\flushright $C_{in}$ \\ $0\sim\ $\SI{47}{\pico\farad}}] (2.5,2)
;
\draw [dashed] (0.55,-0.3) rectangle (4.6,2.25)
;
\end{circuitikz}
\caption{Ricevitore di segnale}
\end{subfigure}
\caption{Circuiti equivalenti}
\end{figure}
