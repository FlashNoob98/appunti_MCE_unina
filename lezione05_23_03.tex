\section{Sorgenti e ricevitori di segnale}
È possibile rappresentare con dei circuiti equivalenti le sorgenti dei disturbi
e il ricevitore di interferenza
\begin{figure}[h]
\centering
\begin{subfigure}[t]{0.4\textwidth}
\begin{circuitikz}
\draw
(0,0) to [sinusoidal voltage source, l=$V_s$] (0,2)
        to [resistor,-o,l=\mbox{$R_s$}, a=\SI{50}{\ohm}] (3.5,2)
(0,0) to [short , -o] (3.5,0)
;
\draw [dashed] (-0.5,-0.3) rectangle (3,2.25)
;
\end{circuitikz}
\caption{Generatore di disturbi}
\end{subfigure}
\ 
\begin{subfigure}[t]{0.4\textwidth}
\begin{circuitikz}
\draw
(0,0) to [open, o-o] (0,2)
        to [short] (3.5,2)
        to [resistor , l=\parbox{2cm}{$R_{in}\\ \SI{50}{\ohm}$}] (3.5,0)
        to [short] (0,0)
(2.5,0)   to [capacitor, l=\parbox{2.3cm}{\flushright $C_{in}$ \\ $0\sim\ $\SI{47}{\pico\farad}}] (2.5,2)
;
\draw [dashed] (0.55,-0.3) rectangle (4.6,2.25)
;
\end{circuitikz}
\caption{Ricevitore di segnale}
\end{subfigure}
\caption{Circuiti equivalenti}
\end{figure}

La capacità presente nel circuito schematico del ricevitore, schematizza la capacità
parassita dovuta al tipo di schermatura del cavo coassiale che può
essere vista come un condensatore cilindrico.
Le indicazioni fornite dal generatore sulla tensione fornita, sono quelle
nel caso di carico adattato a \SI{50}{\ohm}.

% in caso di carico adattato, la tensione sul ricevitore non dipende
% dalla lunghezza del cavo VEDI impedenza caratteristica del cavo
\section{Ricevitore di interferenze}
Per analizzare un segnale di disturbo, è necessario studiarne le componenti
in frequenza, ogni strumento di misura ha un filtro in ingresso
con una determinata ``\textit{Resolution BandWidth}'' (RBW) ossia l'ampiezza
di banda che il filtro è in grado di analizzare; per studiare l'intero segnale
è necessario ``scorrere'' l'intero segnale lungo la sua intera banda.

Sorge un problema, la risoluzione relativa è definita come $\frac{\text{RBW}}{f} $
dove $f$ è la frequenza del segnale analizzato, un filtro con RBW fissata ad esempio 
ad \SI{1}{\kilo\hertz} dovrebbe essere in grado anche di 
analizzare segnali nell'ordine del \SI{}{\giga\hertz}, ma la risoluzione relativa
sarebbe dell'ordine di $10^{-6}$, sicuramente un requisito troppo stringente.

La soluzione a questo problema è stata quella di utilizzare il ricevitore
a \textbf{supereterodina}, piuttosto che spostare la frequenza di lavoro del filtro 
si preferisce traslare il segnale mediante l'utilizzo di un \textbf{mixer}.

La funzione del mixer è quella di ``moltiplicare'' il segnale in ingresso
per un segnale sinusoidale a frequenza fissata, generata con 
un \textit{oscillatore locale},
il risultato sarà una coppia di segnali,
uno con frequenza pari alla somma della frequenza del segnale in ingresso e quella
fornita dall'oscillatore, l'altro pari alla differenza fra le due,
in particolare si riporta la prima formula di Werner che dimostra quanto 
appena affermato:
$$
\sin\alpha\cos\beta = \frac{1}{2} \left[\sin(\alpha+\beta)+\sin(\alpha-\beta)\right]
$$

Si \textit{pilota} dunque la frequenza $f_{\text{lo}}$ dell'oscillatore locale
affinché $f_x - f_{\text{lo}} = f_{\text{if}}$ con $f_{\text{if}}$ pari alla
frequenza del filtro a frequenza intermedia in ingresso.

