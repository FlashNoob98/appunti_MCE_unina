\section{Immunità ai disturbi condotti indotti da campi a RF} (EN 61000-4-6)
Ci si riferisce tipicamente ai valori da \SI{150}{\kilo\hertz} a \SI{80}{\mega\hertz}
o \SI{230}{\mega\hertz} per alcune applicazioni.
Nonostante la dicitura della norma, la misura di immunità avviene iniettando
direttamente il disturbo sui cavi di alimentazione senza la necessità di irradiarli 
e di usare quindi una camera semianecoica, sarebbe infatti difficile accoppiare un disturbo
a radiofrequenza con il cavo di alimentazione, un disturbo a \SI{1}{\mega\hertz}
avrebbe infatti una lunghezza d'onda di circa \SI{300}{\meter}, un cavo di alimentazione
di circa \SI{2}{\meter} sarebbe una pessima antenna. Lo stesso discorso vale
per la lunghezza di una eventuale antenna in emissione.

Il dispositivo che simula il disturbo radiato deve iniettare nei cavi
un disturbo di modo comune, si suppone infatti che il campo di disturbo
induca la corrente sempre nello stesso verso in tutti i cavi.

Anche l'impedenza dell'antenna va simulata, la sorgente ha tipicamente un'impedenza
di \SI{150}{\ohm}.
%Inserisci rappresentazione dispositivo immissione disturbi 0:30:00
\begin{figure}[h]
 \begin{circuitikz}
 \end{circuitikz}
\end{figure}

Si parla di CDN (\textit{Coupling and Decoupling Network}), i condensatori sono necessari a 
richiudere a terra eventuali disturbi provenienti dalla rete, con una
capacità $C_2$ di circa \SI{47}{\micro\farad}, il valore delle induttanze $L$
è maggiore o uguale a \SI{280}{\micro\henry} alla frequenza di \SI{150}{\kilo\hertz}.
Il condensatore $C_1$ è pari a \SI{10}{\nano\farad}, 
