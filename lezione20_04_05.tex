\section{Immunità ai disturbi condotti indotti da campi a RF} (EN 61000-4-6)
Ci si riferisce tipicamente ai valori da \SI{150}{\kilo\hertz} a \SI{80}{\mega\hertz}
o \SI{230}{\mega\hertz} per alcune applicazioni.
Nonostante la dicitura della norma, la misura di immunità avviene iniettando
direttamente il disturbo sui cavi di alimentazione senza la necessità di irradiarli 
e di usare quindi una camera semianecoica, sarebbe infatti difficile accoppiare un disturbo
a radiofrequenza con il cavo di alimentazione, un disturbo a \SI{1}{\mega\hertz}
avrebbe infatti una lunghezza d'onda di circa \SI{300}{\meter}, un cavo di alimentazione
di circa \SI{2}{\meter} sarebbe una pessima antenna. Lo stesso discorso vale
per la lunghezza di una eventuale antenna in emissione.

Il dispositivo che simula il disturbo radiato deve iniettare nei cavi
un disturbo di modo comune, si suppone infatti che il campo di disturbo
induca la corrente sempre nello stesso verso in tutti i cavi.

Anche l'impedenza dell'antenna va simulata, la sorgente ha tipicamente un'impedenza
di \SI{150}{\ohm}.
%Inserisci rappresentazione dispositivo immissione disturbi 0:30:00
\begin{figure}[h] %CDN drawing
\centering
 \begin{circuitikz}[scale=1.2]
 \draw (-0.5,2) node{P};
 \draw (0,2)  to [short,*-] (4,2)
              to [inductor,l=L] (6,2)
              to [short,-*] (9,2);\draw (9.6,2) node{OUT};
              
 \draw (-0.5,1) node{N};
 \draw (0,1)  to [short,*-] (4,1)
              to [inductor,l=L] (6,1)
              to [short,-*] (9,1);
              
              
 \draw (-0.5,0) node{PE};
 \draw (0,0)  to [short,*-] (4,0)
              to [inductor,l=L] (6,0)
              to [short,-*] (9,0);
              
 \draw (3,2) to [capacitor,*-,l=$C_2$] (3,1)
             to [short,-*] (3,0);
             
 \draw (2,1) to [capacitor,*-*,l=$C_2$] (2,0);
 \draw (1,0) node[ground]{};
 
 \draw (6,2) to [resistor,*-,l=R] (6,4);
 \draw (7,2) to [resistor,l=R] (7,4);
 \draw (7,2) to [short,-*] (7,1);
 \draw (8,2) to [resistor,l=R] (8,4);
 \draw (8,2) to [short,-*] (8,0);
 
 \draw (6,4) to [capacitor,l=$C_1$] (6,5)
             to [short] (8,5)
             to [short] (8,4);
 \draw (7,4) to [capacitor,l_=$C_1$] (7,5);
 
 \draw (7.3,5.5) node{IN};
 \draw (7,5) to [short,-*] (7,5.5)
             to [short] (7,6)
             to [resistor,l_=$\SI{50}{\ohm}$] (5.5,6);
 \draw (4,6) to [vsourcesin] (5.5,6);
 \draw (4,6) node[ground]{};
 \end{circuitikz}
 \caption{M3 CDN}
 \label{fig:cdn}
\end{figure}

Si parla di CDN (\textit{Coupling and Decoupling Network}), i condensatori sono necessari a 
richiudere a terra eventuali disturbi provenienti dalla rete, con una
capacità $C_2$ di circa \SI{47}{\micro\farad}, il valore delle induttanze $L$
è maggiore o uguale a \SI{280}{\micro\henry} alla frequenza di \SI{150}{\kilo\hertz}.
Il condensatore $C_1$ è pari a \SI{10}{\nano\farad}.

Il disturbo si ripartisce nello stesso verso su tutti i cavi, per direzionare
il disturbo solo verso l'EUT e non verso la rete, le induttanze si comportano
come circuiti aperti in alta frequenza.
I condensatori $C_1$ servono ad isolare la componente a \SI{50}{\hertz} dal
generatore di disturbi ed evitare eventuali danni.

Esistono vari tipi di CDN in funzione della tipologia di cavo, quella rappresentata
in figura \ref{fig:cdn} è utilizzata per dispositivi monofase con terra di protezione.

Per misurare l'impedenza di modo comune $Z_{\text{CM}}$ si uniscono i 3 cavi terminali 
della CDN e si misura l'impedenza rispetto a terra. Questa impedenza è data dal 
parallelo delle tre resistenze sulla porta di ingresso e dalla somma della
resistenza del generatore di \SI{50}{\ohm}, il valore di ogni resistenza R sarà quindi 
pari a \SI{300}{\ohm}.

Se si suppone di avere una rete M2, ovvero con solo due conduttori e quindi 2 resistenze,
esse avranno un valore di \SI{200}{\ohm}, una rete M1 avrà una resistenza da 
\SI{100}{\ohm} e così via.

Se si dovesse analizzare il disturbo su un cavo schermato contenente 25 pin,
la CDN dovrebbe essere comunque di tipo M1 e iniettare i disturbi soltanto
sullo schermo esterno, non ci sarebbe alcuna possibilità per i disturbi
di raggiungere i conduttori interni allo schermo.
%0:57
\paragraph{Taratura della CDN}
La taratura della \textit{CDN} viene eseguita mediante la misurazione della
tensione di modo comune d'uscita a vuoto, come indicato dalle norme, tipicamente
\SI{3}{\volt_{RMS}} .
Non si può collegare direttamente l'EMI receiver all'uscita della CDN a causa
della sua impedenza di \SI{50}{\ohm} che si comporterebbe  come un carico
disadattato per la CDN, generando una riflessione di potenza verso il
generatore. Si interpone quindi un dispositivo di adattamento con una impedenza
equivalente di \SI{100}{\ohm}. Un ulteriore dispositivo di adattamento è interposto 
tra il generatore e la CDN, un tronco di linea di trasmissione dissipativo con
attenuazione da \SI{6}{\decibel} e impedenza caratteristica di \SI{50}{\ohm}.

Il segnale $V_{\text{out}}$ in uscita dal generatore
verrà quindi attenuato a $V_{\text{out}}/2$ e vedrà un'impedenza
di \SI{250}{\ohm} dall'ingresso della CDN in poi, ci sarà un'onda
riflessa pari a $\Gamma\cdot V_{\text{out}}/4$ dopo aver ripercorso la linea 
di trasmissione da \SI{6}{\decibel}. Essendo $\Gamma$ sicuramente minore di 1,
si può considerare trascurabile l'ampiezza dell'onda riflessa e ritenere tutto
sommato il carico adattato.

La linea di attenuazione di \SI{6}{\decibel} verrà quindi utilizzata anche in fase 
operativa.

Dato che la tensione all'uscita della CDN si ha con un carico di \SI{150}{\ohm} e 
una sorgente di pari impedenza, essa sarà pari alla metà della tensione a 
vuoto $V_0$, ossia $V_0' = V_0/2$. La tensione misurata dal EMI receiver
$V_m$ sarà la partizione di $V_0'$ sulla resistenza interna dell'EMI, ossia
$V_m = V_0' \cdot \frac{50}{100+50} = V_0'/3 = V_0/6$
Si deve quindi regolare la tensione del generatore $V_G$ affinchè si misuri
la tensione target, al variare della frequenza però possono crearsi accoppiamenti
capacitivi parassiti che richiederanno una tensione del generatore superiore
per ottenere la stessa $V_0$, la norma concede però una tolleranza
di $\pm 25\%$ sul valore nominale di $V_0$.

Anche in questo caso, con l'immissione dei disturbi sul cavo di alimentazione
si effettuerà una modulazione in ampiezza del segnale con una frequenza
di \SI{1}{\kilo\hertz} e una profondità di modulazione dell'80\% in analogia
allo studio dell'immunità radiata.
