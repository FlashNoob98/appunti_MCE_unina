\section{Analizzatore di spettro o ricevitore di interferenza?}
Si elencano le principali differenze tra i due strumenti principali
utilizzati nella misura dei disturbi: l'\textbf{analizzatore di spettro} e il \textbf{ricevitore di interferenze}.

\begin{center}%tabella emi receiver vs spectrum analyzer
\begin{tabular}{|>{\centering}m{4cm}|>{\centering}m{4cm}|m{4cm}<{\centering}|}
 \rowcolor{blue}
 \hline
  \color{white}\textbf{Parametro}&\color{white}\textbf{Spectrum analyzer} &\color{white} \textbf{EMI receiver} \\
  \hline
  Principio di funzionamento & Spazzolata (sweeped) & Frequenze discrete (\textit{tuned}) \\
  \hline 
  RBW [\si{\kilo\hertz}] &1, 3, 10, 30, 100, ... & 0.2, 9, 120 \\
  \hline
  Banda & \SI{3}{\decibel} & \SI{6}{\decibel} \\
  \hline
  Rivelatore & Pk & (Pk), QP, AVG\\
  \hline
  Selettività e preamplificazione & $-$ & $+$ \\
  \hline
  Accuracy (A \& f) & $-$ & $+$ \\
  \hline
  Resistenza all'overload & $\sim$ & $+$ \\
  \hline
  Costo & $-$ & $++$ \\
  \hline
  Applicazioni & Pre compliance & Full compliance \\
  \hline
\end{tabular}
\end{center}

La definizione di \textit{banda} per i due strumenti è differente,
nell'\textit{analizzatore di spettro} la \textit{resolution bandwidth} 
è l'ampiezza dell'intervallo i cui valori vengono attenuati ad una differenza di massimo 
\SI{3}{\decibel} dal valore
del segnale alla frequenza centrale del filtro.
Questo valore nel \textit{ricevitore di interferenze} è pari a \SI{6}{\decibel}.

L'analizzatore di spettro esegue un'analisi rapida, riportando solo
il valore di picco dell'inviluppo. L'EMI receiver
dispone anche dei rilevatori di QuasiPicco ed Average, il rivelatore
di Picco non è necessario al fine delle verifiche in campo civile.

La \textbf{selettività} indica quanto rapidamente un filtro tende
a 0, anche se 2 filtri hanno la stessa RBW, i valori al di fuori della banda
possono attenuarsi con pendenze diverse.
La \textbf{preamplificazione} è necessaria alla corretta analisi di segnali
di disturbo ad una determinata frequenza che, se di ampiezza bassa, possono
essere confusi con il rumore di fondo dal rilevatore.

L'\textbf{accuracy} è maggiore nell'EMI receiver.
La resistenza all'\textbf{overload} è la resistenza al sovraccarico
degli stadi di ingresso, l'analizzatore di spettro non prevede
segnali in ingresso distruttivi dato che viene prevalentemente usato in 
ambienti controllati o comunque con segnali noti, cosa non vera
per l'EMI receiver, con il quale si misurano disturbi di cui non 
conosciamo preventivamente l'entità.

Il costo di un EMI receiver è molto più alto di quello
di un analizzatore di spettro.
L'analizzatore di spettro permette l'esecuzione
di prove pre-compliance, utili ad avere un'idea dei
possibili problemi di compatibilità, per aver la conformità
alla direttiva è invece necessario il ricevitore di interferenze.

Come si è visto la differenza principale tra i due strumenti consta nella 
selettività e nella preamplificazione, nell'EMI receiver
infatti, in aggiunta a quanto presente anche nell'analizzatore di spettro
è necessario aggiungere un \underline{attenuatore} contro gli 
overload e un \underline{filtro passabanda} necessario a limitare la potenza
in ingresso al mixer per ridurre al minimo gli errori dovuti alla non linearità.
Deve essere inoltre presente un \underline{preamplificatore} necessario ad
innalzare il valore di segnali particolarmente bassi.

Il \underline{preselettore} deve lavorare su ampie gamme di frequenze e non
è richiesto che sia molto stretto, deve solo limitare l'energia
in ingresso al mixer, non svolge alcuna funzione di misura.

\paragraph{Rivelatore di Picco e QuasiPicco}
Il seguente circuito rappresenta uno schema del rivelatore di Picco o di QuasiPicco,
la differenza tra i due rivelatori si ottiene modificando opportunamente la resistenza
di scarica $R_d$.

\begin{figure}[h] %rivelatore di picco
\centering
 \begin{circuitikz}[american voltages]
 \draw
 (0,2) to [full diode,o-,v=$V_d$] (2,2)
       to [resistor,l=$R_c$] (4,2)
       to [capacitor,l=$C$] (4,0)
 (4,2) to [short,-o] (7,2)   
 (6,2) to [resistor,l=$R_d$] (6,0)
 (0,0) to [short, o-o] (7,0)
 ;
 \draw
 (-0.5,2) to [open, v=$V_{\text{in}}$] (-0.5,0)
 (7.6,2) to [open, v^>=$V_{\text{out}}$] (7.6,0)
 ;
 \end{circuitikz}
 \caption{Rivelatore di picco}
\end{figure}

Il rivelatore di Picco ha una resistenza di scarica $R_d$ molto elevata,
il rivelatore di quasipicco invece ha una resistenza di valore confrontabile
con quello della resistenza di carica $R_c$.

Bisogna tenere in considerazione il tempo di misura affinché i valori
misurati, rispettivamente di picco o quasipicco, siano
effettivamente i massimi valori assunti dal segnale e bisogna assicurarsi che
i rivelatori siano andati a regime, i tempi di misura per ogni frequenza possono
essere anche dell'ordine delle centinaia di millisecondi.
